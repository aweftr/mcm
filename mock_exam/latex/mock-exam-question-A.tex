\documentclass{cumcm}


% \title{text}这里是显示在第三页的文章标题
\title{锂电池SOH预测模型设计}
% \displaytitle{text} 这里是显示在承诺书上的文章标题,注意,不能换行,如果题目特别长,要进行适当的缩写
\displaytitle{锂电池SOH预测模型设计}
% \school{text}命令用于在承诺书上显示学校名称。按要求,此处应填写全称
\school{上海交通大学}
% 以下命令分别显示队员、指导教师姓名以及队伍编号
\authorone{程云龙}
\authortwo{李羚子}
\authorthree{李汉威}
\advisor{高晓沨}
\teamnumber{130}

\begin{document}

% 这里用于打印承诺书以及编号页
 \input{declaration.tex}

\begin{minipage}{0.9\textwidth}
\centering\LARGE\textbf{锂电池SOH预测模型设计}
\end{minipage}
\begin{abstract}
本题题意在于通过对测试数据充足的锂电池建立SOH模型,找出可测数据与电池SOH之间的关系,并使模型适用于不同的锂电池,进而完成测试数据较少的锂电池SOH预测。\par
\textbf{针对问题一}\quad 我们通过采用电化学阻抗谱法和电压曲线拟合法,分别找出电池电阻、恒流充电电压与电池SOH之间的关系,从而建立预测模型。然后利用层次分析法,找出内阻与电压两个因素对电池SOH影响所占权重,加权计算出


\textbf{关键词} \quad MATLAB \quad 电化学阻抗谱法  \quad 电压曲线拟合法
\end{abstract}

\newpage
\section{问题重述}
\subsection{问题背景}
电池是一种能量转化与储存的装置,它通过反应可将化学能或物理能转化为电能,从而得到具有稳定电压、长时间稳定供电的电流。作为众多移动终端的动力单元及绿色环保电池的首选,锂离子电池在电池界具有举足轻重的地位,并在各种电子设备中得到了越来越广泛的应用。其在便携式设备、储备电源、卫星、电动汽车等领域,更是具有广阔的前景。
\subsection{问题的提出(题目重述)}
随着使用情况的增多,锂离子电池会逐渐老化,我们通常采用SOH(锂电池放电容量与锂电池额定容量的比值)表示锂离子电池的老化程度。为了保证设备的稳定,预测锂电池的SOH便具有了重要的意义。然而不同类型的锂电池,其测试数据的多少会有不同。要求利用一个测试数据充足的锂电池SOH预测模型,通过数学建模方法转化,完成测试数据较少的锂电池SOH预测。

\begin{itemize}
\item \textbf{任务一} \quad 利用源数据中B0007号锂电池全部循环测试的充放电参数(电压、电流等)建模预测源数据中B0005号锂电池SOH,计算真实SOH值与预测值的均方根误差。
\item \textbf{任务二} \quad 取cell1、cell3、cell7、cell8中前一半数据,结合任务一B0007号锂电池的模型再次建模,对这四种电池中另一半缺失数据的锂电池SOH进行预测,计算真实SOH值与预测值的均方根误差。
\end{itemize}

\section{模型假设}
\begin{enumerate}
\item 在本题所给数据中,环境温度恒为$24^\circ$C,所以外界环境温度对电池SOH的影响不予考虑。
\item 不考虑恒压充电情况。
\item 不考虑充放电电流对电池SOH值影响。
\item 将电池EIS、恒流充电电压作为电池SOH的两大影响因素。
\end{enumerate}
\section{假设说明}
\begin{enumerate}
\item EIS为电池老化过程中阻抗谱变化。
%\item 由参考文献\cite{}可知,在恒压充电情况下,电池SOH值改变较小,可忽略,仅考虑恒流充电情况。
\item 考虑电池在放电过程中,放电电流、温度等因素较为复杂,而充电过程情况较为稳定,因此选用电池充电电压曲线拟合估算。
%\item 由论文\cite{}可知,电池EIS会随着电池老化而发生相应改变,说明EIS变化与电池SOH衰减具有相关性。
\end{enumerate}


\section{问题分析}
对于问题一和问题二,可采用同样的方法进行建模。\par
由题目所给测试数据及所查参考文献,对于锂电池老化机理分析,我们仅考虑电池EIS及恒压充电电压两个因素的改变对电池SOH值的影响。%论文\cite{}
对阻抗参数与电池老化规律进行了分析,并验证了基于阻抗法估算电池SOH的可行性。通过电化学阻抗谱模型可以建立电池EIS与电池老化过程中相应内部参数之间的联系;通过电压曲线拟合法得出充电电压与电池SOH的关系。然后利用层次分析法,找出两因素的影响权重,加权计算得到最终的SOH预测结果,算出其与真实SOH的均方根误差,进行比较。
\subsection{问题一分析}
\begin{enumerate}[(1)]
\item \textbf{电化学阻抗谱法}\par
本题所给测试数据中,采用循环充放电方式对BOOO7号样本电池老化试验,我们可得出随着循环次数的增加,电池SOH衰减曲线,如图所示:
\begin{figure}[H]
\centering
\includegraphics[width=0.8\textwidth]{img/7-SOH-cycle.png}
\caption{电池B0007的SOH与循环次数关系曲线}\label{figure-a}
\end{figure}
在循环次数内,SOH值基本呈线性下降,属正常老化。同时,电池内阻随着循环次数增加而逐步增加,可得到如图所示结果。
\begin{figure}[H]
\centering
\includegraphics[width=0.8\textwidth]{img/R-cycle.png}
\caption{电池B0007的内阻与循环次数关系曲线}\label{figure-b}
\end{figure}

我们发现在电池循环前期,随着循环次数的增加,电池电阻缓慢下降,当循环试验进行到后期。随着循环次数的增加,电池内阻会有很大程度的下降。其中,前面波动较大是由于新电池刚开始充电时性能不稳定造成的,所以可把这一部分不考虑在内,只考虑循环后期电池状况稳定的情况。对余下曲线进行拟合,如图所示:\par
\begin{figure}[H]
\centering
\includegraphics[width=0.8\textwidth]{img/R-cycle-fit.png}
\caption{电池B0007的内阻与循环次数的拟合曲线}
\end{figure}
所以,所给数据符合电化学阻抗谱法适用条件。

\item \textbf{电压曲线拟合法}
\end{enumerate}
\subsection{问题二分析}



\section{问题的解答}
\subsection{问题1的解答}
\begin{enumerate}[(1)]
\item \textbf{电化学阻抗谱法}
我们用电阻测试结果表示SOH,可获得如图的SOH和电池电阻关系图,分析数据可知,电池SOH随着电阻老化呈单调下降,采用四次拟合的方法,我们可近似计算电池SOH老化规律曲线
\begin{figure}[H]
\centering
\includegraphics[width=0.8\textwidth]{img/SOH-R-fit.png}
\caption{电池B0007的SOH与其内阻变化曲线}
\end{figure}
通过MATLAB拟合求解,可得到相关系数为$0.9483$,表明两者相关性较强,可用该模型来预测B0005电池SOH。代入目标数据求得B0005预测SOH与真实值之间的均方根误差为$0.0296j$。
\item \textbf{电压曲线拟合法}

\end{enumerate}
\subsection{问题2的解答}


\section{模型总结}

\subsection{模型优点}

\subsection{模型缺点}


\bibliographystyle{plain}
\bibliography{ref}

\newpage
\appendix
\textbf{附录}
\section{模型求解代码}

\end{document}

