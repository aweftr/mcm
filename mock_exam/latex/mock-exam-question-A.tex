\documentclass{cumcm}


% \title{text}这里是显示在第三页的文章标题
\title{}
% \displaytitle{text} 这里是显示在承诺书上的文章标题,注意,不能换行,如果题目特别长,要进行适当的缩写
\displaytitle{}
% \school{text}命令用于在承诺书上显示学校名称。按要求,此处应填写全称
\school{上海交通大学}
% 以下命令分别显示队员、指导教师姓名以及队伍编号
\authorone{}
\authortwo{}
\authorthree{}
\advisor{}
\teamnumber{}

\begin{document}

% 这里用于打印承诺书以及编号页
 \input{declaration.tex}

\begin{minipage}{0.9\textwidth}
\centering\LARGE\textbf{}
\end{minipage}
\begin{abstract}
% \begin{abstract} 摘要

% \textbf{关键词} 关键词
\end{abstract}

\newpage
\section{问题重述}
\subsection{问题背景}
电池是一种能量转化与储存的装置,它通过反应可将化学能或物理能转化为电能,从而得到具有稳定电压、长时间稳定供电的电流。作为众多移动终端的动力单元及绿色环保电池的首选,锂离子电池在电池界具有举足轻重的地位,并在各种电子设备中得到了越来越广泛的应用。其在便携式设备、储备电源、卫星、电动汽车等领域,更是具有广阔的前景。
\subsection{问题的提出(题目重述)}
随着使用情况的增多,锂离子电池会逐渐老化,我们通常采用SOH(锂电池放电容量与锂电池额定容量的比值)表示锂离子电池的老化程度。为了保证设备的稳定,预测锂电池的SOH便具有了重要的意义。然而不同类型的锂电池,其测试数据的多少会有不同。要求利用一个测试数据充足的锂电池SOH预测模型,通过数学建模方法转化,完成测试数据较少的锂电池SOH预测。

\begin{itemize}
\item \textbf{任务一} \quad 利用源数据中B0007号锂电池全部循环测试的充放电参数(电压、电流等)建模预测源数据中B0005号锂电池SOH,计算真实SOH值与预测值的均方根误差。
\item \textbf{任务二} \quad 取cell1、cell3、cell7、cell8中前一半数据,结合任务一B0007号锂电池的模型再次建模,对这四种电池中另一半缺失数据的锂电池SOH进行预测,计算真实SOH值与预测值的均方根误差。
\end{itemize}

\section{模型假设}
\begin{enumerate}
\item 在本题所给数据中,环境温度恒为$24^\circ$C,所以外界环境温度对电池SOH的影响不予考虑。
\item 不考虑恒压充电情况。
\item 不考虑充放电电流对电池SOH值影响。
\item 将电池EIS、充电电压作为电池SOH的两大影响因素。
\end{enumerate}
\section{假设说明}
\begin{enumerate}
\item EIS为电池老化过程中阻抗谱变化。
\item 由参考文献可知,在恒压充电情况下,电池SOH值改变较小,可忽略,仅考虑恒流充电情况。
\item 考虑电池在放电过程中情况比较复杂,放电电流、温度等因素较为复杂,而充电过程情况较为稳定,因此选用电池充电电压曲线估拟合估算。
\item 由数据分析可知,恒压情况下电池SOH改变可忽略不计。
 由论文可知,电池EIS会随着电池老化而发生相应改变,说明EIS变化与电池SOH衰减具有相关性。
\end{enumerate}
\section{符号说明}

\section{问题分析}
对于问题一和问题二,我们可采用同样的方法进行建模。\par
由题目所给测试数据及所查参考文献,对于锂电池老化机理分析,我们仅考虑电池EIS及充电电压两个因素的改变对电池SOH值的影响。通过电化学阻抗谱模型可以建立电池EIS与电池老化过程中相应内部参数之间的联系;通过电压曲线拟合法得出充电电压与电池SOH的关系。然后利用层次分析法,找出两因素的影响权重,加权计算得到最终的SOH预测结果,算出其与真实SOH的均方根误差,进行比较。


\section{问题的解答}
\subsection{问题1的解答}

\subsection{问题2的解答}


\section{模型总结}

\subsection{模型优点}

\subsection{模型缺点}


\bibliographystyle{plain}
\bibliography{ref}

\newpage
\appendix
\textbf{附录}
\section{模型求解代码}

\end{document}

