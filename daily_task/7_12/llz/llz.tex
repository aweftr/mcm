\documentclass[12pt,a4paper]{article}

\title{MCM}
\author{llz}
\date{\today}
\usepackage{enumerate}
\usepackage{algorithm2e}
\usepackage{amsmath}
\usepackage[UTF8]{ctex}
\usepackage{gensymb}

\begin{document}
\kaishu
\maketitle

\section{说明}
周末两天会离开伯克利去别的城市做field work,这两天的内容下周会补上

\section{存贮模型}
\subsection{不允许缺货的确定性贮存模型}
\subsubsection{模型假设}
\begin{enumerate}
\item在任何时刻,单位时间(每天)对物品的需求量恒为$r$吨;即经营商品单一,顾客对该物品需求量在时间上保持恒定;
\item每隔时间$T$天进货$Q$吨;且假设每次进货是在存货全部售出后即刻进行,不允许缺货,即$Q=rT$;
\item每次进货需支付订货费$c_1$,在正常期间,还需支付货物的贮存费用,单位时间(天)单位(吨)货物需支付货物的贮存费用$c_2$;
\item 以$q(t)$表示在时刻$t$该货物的存量
\end{enumerate}
\subsubsection{模型建立}
 $$Min:C=\frac{(c_1+c_2\int_{0}^{T}q(t)dt)}{T}$$\\
\begin{displaymath}
\begin{split}
 s.t \quad q(t)&=Q-rt\\
 Q&=rT
\end{split}
\end{displaymath}
进一步简化得到:\\
$$Min: C=\frac{c_1}{T}+\frac{C_2 rT}{2}$$
\subsubsection{模型求解}
令$\frac{dC}{dT}=0$,得最优进货周期$T=\sqrt{\frac{2c_1}{rc_2}}$,进而每次的进货量$Q=rT=\sqrt{\frac{2rc_1}{c_2}}$\\
即经济订货批量公式

\subsubsection{模型点评}
从模型的解可以发现,当订货费越高,需求量越大时,一次订货量应越大;当贮存费越高,一次订货量应越小。这些关系是符合常识的,但仅凭常识是不能得到准确的依从关系。



\end{document}
