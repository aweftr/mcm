\documentclass[12pt,a4paper]{article}

\usepackage[UTF8]{ctex}
\usepackage{amsmath,amscd,amsbsy,amssymb,latexsym,url,bm,amsthm}
\usepackage{amsfonts}
\usepackage{epsfig,graphicx,subfigure}
\usepackage{hyperref}
\usepackage{listings}
\usepackage[vlined,ruled,linesnumbered]{algorithm2e}
\usepackage{enumitem}
\usepackage{xcolor}
\usepackage{geometry}

%\uppercase\expandafter{\romannumeral1}:% 罗马数字。

\lstset{
language=Matlab,
keywordstyle= \color{blue!70},
commentstyle= \color{red!50!green!50!blue!50},
breaklines
}%设置listing插入语言

\setlength{\parindent}{0em}
\setlength{\parskip}{1em}

\geometry{bottom =3cm}
\newcommand{\textbi}[1]{%
\textbf{\textit{#1}}}

\newcommand{\ncolor}[1]{%
{\color[RGB]{139,117,0}{#1}}}
\newtheorem{theorem}{Theorem}[section]
\newenvironment{solution}{{\noindent \it \textbf{Solution:}}\\}

\title{MCM daily}
\author{Yunlong Cheng}

\begin{document}
\maketitle
\section{粒子群 (PSO) 算法}
\subsection{初识 PSO 算法}
\begin{enumerate}
  \item 属于软计算。
  \item 对群体行为有简单的准则:
  \begin{itemize}
    \item 冲突避免
    \item 速度匹配
    \item 群体中心
  \end{itemize}
\end{enumerate}
\subsection{PSO 算法的基本理论}
在1995年基于群鸟觅食提出。

每个个体看成一个粒子,鸟群看成一个粒子群。在 $D$ 维目标搜索空间中,有 $m$ 个粒子组成一个群体,每个粒子的位置就是一个潜在解,将 $X_i$ 带入目标函数计算适应值,根据适应值的大小衡量其优劣。粒子个体经历过的最好位置为 $P_i$,整个群体所有粒子经历过的最好位置记为 $P_g$。粒子 $i$ 的速度记为 $V_i$。

采用下列公式对例子所在位置不断更新:
$$v_i^d = \omega v_i^d + c_1r_1(p_i^d - x_i^d) + c_2r_2(p_g^d - x_i^d)$$
$$x_i^d = x_i^d + \alpha v_i^d$$
其中 $\omega$ 为惯性因子,非负,$c_1,c_2$ 为加速常数,非负,$r_1,r_2$为[0,1]范围内变换的随机数,$\alpha$为约束因子,控制速度权重。
\subsection{约束优化}
\begin{itemize}
  \item 罚函数法
  \item 将搜索范围限制在条件约束簇内。
\end{itemize}
\subsection{优缺点}
优点:
\begin{itemize}
  \item 容易飞跃局部最优信息。
  \item 调整参数少,原理简单。
\end{itemize}
缺点:
\begin{itemize}
  \item 搜索精度不高。
  \item 不能保证搜索到全局最优解。
  \item 算法理论不完善,无数学方面严格证明。
\end{itemize}
\subsection{程序设计}
\end{document}
