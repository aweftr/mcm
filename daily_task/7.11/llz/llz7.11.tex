\documentclass[12pt,a4paper]{article}

\title{MCM}
\author{llz}
\date{\today}
\usepackage{enumerate}
\usepackage{algorithm2e}
\usepackage{amsmath}
\usepackage[UTF8]{ctex}
\usepackage{gensymb}

\begin{document}
\kaishu
\maketitle

\section{双层玻璃窗功效}
\subsection{模型假设}
\begin{enumerate}
\item{热量传播形式只有传导,没有对流}
\item{室内温度$T_1$和室外温度$T_2$保持不变}
\item{玻璃材料均匀,热传导系数是常数$K_1$,空气的热传导系数是常数$K_2$}
\end{enumerate} 
\subsection{模型建立}
由物理定律:单位时间由温度高的一侧向温度低的一侧通过单位面积的热量,与温度差成正比,与距离成反比:
\begin{equation}
Q=K_1\frac{T_1-T_a}{d}=K_2\frac{T_a-T_b}{l}=k_1\frac{T_b-T_2}{d}
\end{equation}
d、l分别表示玻璃以及中间夹层的厚度\\
$T_a$、$T_b$表示中间夹层两侧温度\\
求解可得:
\begin{equation}
Q=\frac{k_1k_2}{lk_1+2dk_2}(T_1-T_2)
\end{equation}


\end{document}