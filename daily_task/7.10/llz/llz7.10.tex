\documentclass[12pt,a4paper]{article}

\title{MCM}
\author{llz}
\date{\today}
\usepackage{enumerate}
\usepackage{algorithm2e}
\usepackage{amsmath}
\usepackage[UTF8]{ctex}
\usepackage{gensymb}

\begin{document}
\kaishu
\maketitle

\section{公平席位分配}
\subsection{两方席位分配}
\subsubsection{建立数量指标}
\noindent标准一:绝对不公平指标:$\frac{P_1}{n_1}-\frac{P_2}{n_2}$\\
标准二:相对不公平指标:$r_A(n_1,n_2)=\frac{\frac{P_1}{n_1}-\frac{P_2}{n_2}}{\frac{P_2}{n_2}}$,称之为相对于$B$对$A$的相对不公平值\\
\emph{确定席位分配方案的原则是使他们尽可能小}

\subsubsection{确定分配方案}
\begin{enumerate}
\item 若$\frac{P_1}{n_1+1}>\frac{P_2}{n_2}$.则增加席位给A
\item 若$\frac{P_1}{n_1}>\frac{P_2}{n_2},\frac{P_1}{n_1+1}<\frac{P_2}{n_2}$,计算$r_a(n_1,n_2+1)$,$r_B(n_1+1,n_2)$,若$r_B(n_1+1,n_2)<r_a(n_1,n_2+1)$,则分配给$A$,反之分配给$B$.
\end{enumerate}
\underline{分配原则时使相对不公平值尽可能小}

\subsection{$Q$-值法与$m$方席位的分配}
设第$i$方人数为$P_i$,已占有$n_i$席,当总席数增加1时,计算$Q_i=\frac{P_i^2}{n_i(n_i+1)}$,则增加的1席分配给Q值大的一方

\end{document}