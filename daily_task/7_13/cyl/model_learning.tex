\documentclass[12pt,a4paper]{article}

\usepackage[UTF8]{ctex}
\usepackage{amsmath,amscd,amsbsy,amssymb,latexsym,url,bm,amsthm}
\usepackage{amsfonts}
\usepackage{epsfig,graphicx,subfigure}
\usepackage{hyperref}
\usepackage{listings}
\usepackage[vlined,ruled,linesnumbered]{algorithm2e}
\usepackage{enumitem}
\usepackage{xcolor}
\usepackage{geometry}

%\uppercase\expandafter{\romannumeral1}:% 罗马数字。

\lstset{
language=Matlab,
keywordstyle= \color{blue!70},
commentstyle= \color{red!50!green!50!blue!50},
breaklines
}%设置listing插入语言

\setlength{\parindent}{0em}
\setlength{\parskip}{1em}

\geometry{bottom =3cm}
\newcommand{\textbi}[1]{%
\textbf{\textit{#1}}}

\newcommand{\ncolor}[1]{%
{\color[RGB]{139,117,0}{#1}}}
\newtheorem{theorem}{Theorem}[section]
\newenvironment{solution}{{\noindent \it \textbf{Solution:}}\\}

\title{MCM daily}
\author{Yunlong Cheng}

\begin{document}
\maketitle
\section{PSO 算法后续}
\subsection{参数选取}
\begin{enumerate}
  \item 粒子数 $m$ 一般取 $20\sim 40$,粒子数越多额容易找到全局最优解,运行时间越长。
  \item 惯性因子 $\omega$,越大容易失去局部寻优功能,而全局搜索能力越强。可以是变量,在迭代过程中逐渐减小,也可以是常量。
  \item 加速常数 $c_1$ 和 $c_2$,一般都取2.0。
  \item 最大飞翔速度 $v_{max}$,通常 $v_{max}$ 设置为每维变化范围的 $10\%\sim 20\%$
\end{enumerate}
\section{基于 $PSO$ 算法和 $BP$ 算法训练神经网络}
$BP$ 神经网络容易陷入局部最小值。先将网络用 $PSO$ 算法训练,再用 $BP$ 算法接着进行小范围精细搜索。
\section{模拟退火算法(SA)}
\subsection{概述与基本思想}
\begin{itemize}
  \item 是一种通用概率算法。
  \item 源于固体的退火过程。
  \item 1953年Metropolis提出重要性采样的方法--Metropolis准则。
\end{itemize}
\subsection{参数说明}
\begin{itemize}
  \item 由一组初始参数,即冷却进度表(cooling schedule)控制。
  \item 核心是尽量使系统达到准平衡。
  \item 进度表包括:
  \begin{enumerate}
    \item $T_0$:冷却开始的温度。
    \item 控制参数 $T$ 的衰减函数。
    \item 控制参数 $T$ 的终值 $T_f$。
    \item Markov 链的长度 $L_k$:任意温度 $T$ 的迭代次数。
  \end{enumerate}
\end{itemize}
\subsection{基本步骤}
\begin{enumerate}
  \item 令 $T = T_0$,随机生成一个初始解 $x_0$,计算目标函数 $E(x_0)$。
  \item 令 $T$ 等于冷却进度表中的下一个值 $T_i$。
  \item 根据当前解 $x_i$ 进行扰动,产生一个新解 $x_j$ 计算相应目标函数值。
  \item 若 $\Delta E<0$,则新解 $x_j$ 被接受,作为新的当前解;若 $\Delta E>0 $,则新解 $x_j$ 按概率 $e^{-\frac{\Delta E}{T_i}}$接受。
  \item 再温度 $T_i$ 下,重复 $L_k$ 次扰动和接受过程。
  \item 判断 $T$ 是否到达 $T_f$。
\end{enumerate}
 \textbf{注意:}虽然在低温时接收函数很小,但不排除有接受更差的解的可能,因此一般会把退火过程中碰到的最好的可行解记录下来,与终止算法前最后一个接受解一并输出。
\subsection{几点说明}
\begin{enumerate}
  \item 状态表达。
  \item 新解产生。
  \item 收敛的一般性条件。
  \item 参数选择
  \begin{enumerate}
    \item $T_0$
    \item 控制参数 $T$ 的衰减函数。
    \item Markov 链长度。
  \end{enumerate}
\end{enumerate}
\end{document}
