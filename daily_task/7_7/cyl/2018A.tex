\documentclass[12pt,a4paper]{article}

\usepackage[UTF8]{ctex}
\usepackage{amsmath,amscd,amsbsy,amssymb,latexsym,url,bm,amsthm}
\usepackage{amsfonts}
\usepackage{epsfig,graphicx,subfigure}
\usepackage{hyperref}
\usepackage{listings}
\usepackage[vlined,ruled,linesnumbered]{algorithm2e}
\usepackage{enumitem}
\usepackage{xcolor}
\usepackage{geometry}

%\uppercase\expandafter{\romannumeral1}:% 罗马数字。

\lstset{
language=Matlab,
keywordstyle= \color{blue!70},
commentstyle= \color{red!50!green!50!blue!50},
breaklines
}%设置listing插入语言

\setlength{\parindent}{0em}
\setlength{\parskip}{1em}

\geometry{bottom =3cm}
\newcommand{\textbi}[1]{%
\textbf{\textit{#1}}}

\newcommand{\ncolor}[1]{%
{\color[RGB]{139,117,0}{#1}}}
\newtheorem{theorem}{Theorem}[section]
\newenvironment{solution}{{\noindent \it \textbf{Solution:}}\\}

\title{MCM daily}
\author{Yunlong Cheng}

\begin{document}
\maketitle
\section{高温作业专用服装设计}
\subsection{问题重述}
高温作业服共有三层,其中第1层和外界环境接触,第3层与皮肤之间为第4层。解决下列问题:
\begin{enumerate}
  \item 服装参数由附件1给出,针对数据建立数学模型,计算温度分布。
  \item 环境温度为65°C、第四层的厚度为5.5 mm 时,确定第2层最优厚度,确保工作60分钟,假人皮肤外侧温度不超过47°C,且超过44°C的时间不超过5分钟。
  \item 当环境温度为80°C时,确定第二层和第四层的最优厚度,确保工作 30 分钟时,假人皮肤外侧温度不超过47°C,且超过44°C的时间不超过5分钟。
\end{enumerate}

\section{问题分析与建模}
\subsection{问题分析}
\begin{enumerate}
  \item 问题一:本质就是建立热传导方程,其中要意识到实验室环境和第一层之间以及第四层和皮肤之间存在\textbf{对流换热}。通过给定的温度数据计算相应的对流换热系数$h_1,h_2$。确定热传导方程组。对$h_1$赋值,确定最佳情况的$h_1$,进而确定$h_2$。
  \item 问题二:防热服应该尽可能地轻便,节约材料,而问题二给出了第四层的厚度,其实际上就是单变量优化的问题,求第二层的最小厚度。
  \item 问题三:该问要考虑第二层和第四层的厚度,是双变量优化问题,但是在现实生活中,第四层不影响研发成本,所以主要优化在于减小第二层的厚度。注意:第三问是判断国赛名次的重要问题,要有自己的创新方法。通过大范围枚举搜索估算两个厚度符合条件的范围,再利用小步长搜索符合条件,得到最优解。
\end{enumerate}
\subsection{模型建立}

\end{document}
