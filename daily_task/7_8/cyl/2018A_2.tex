\documentclass[12pt,a4paper]{article}

\usepackage[UTF8]{ctex}
\usepackage{amsmath,amscd,amsbsy,amssymb,latexsym,url,bm,amsthm}
\usepackage{amsfonts}
\usepackage{epsfig,graphicx,subfigure}
\usepackage{hyperref}
\usepackage{listings}
\usepackage[vlined,ruled,linesnumbered]{algorithm2e}
\usepackage{enumitem}
\usepackage{xcolor}
\usepackage{geometry}

%\uppercase\expandafter{\romannumeral1}:% 罗马数字。

\lstset{
language=Matlab,
keywordstyle= \color{blue!70},
commentstyle= \color{red!50!green!50!blue!50},
breaklines
}%设置listing插入语言

\setlength{\parindent}{0em}
\setlength{\parskip}{1em}

\geometry{bottom =3cm}
\newcommand{\textbi}[1]{%
\textbf{\textit{#1}}}

\newcommand{\ncolor}[1]{%
{\color[RGB]{139,117,0}{#1}}}
\newtheorem{theorem}{Theorem}[section]
\newenvironment{solution}{{\noindent \it \textbf{Solution:}}\\}

\title{MCM daily}
\author{Yunlong Cheng}

\begin{document}
\maketitle
\section{高温作业专用服装设计}
\subsection{问题重述}
高温作业服共有三层,其中第1层和外界环境接触,第3层与皮肤之间为第4层。解决下列问题:
\begin{enumerate}
  \item 服装参数由附件1给出,针对数据建立数学模型,计算温度分布。
  \item 环境温度为65°C、第四层的厚度为5.5 mm 时,确定第2层最优厚度,确保工作60分钟,假人皮肤外侧温度不超过47°C,且超过44°C的时间不超过5分钟。
  \item 当环境温度为80°C时,确定第二层和第四层的最优厚度,确保工作 30 分钟时,假人皮肤外侧温度不超过47°C,且超过44°C的时间不超过5分钟。
\end{enumerate}

\section{问题分析与建模}
\subsection{问题分析}
\begin{enumerate}
  \item 问题一:本质就是建立热传导方程,其中要意识到实验室环境和第一层之间以及第四层和皮肤之间存在\textbf{对流换热}。通过给定的温度数据计算相应的对流换热系数$h_1,h_2$。确定热传导方程组。对$h_1$赋值,确定最佳情况的$h_1$,进而确定$h_2$。
  \item 问题二:防热服应该尽可能地轻便,节约材料,而问题二给出了第四层的厚度,其实际上就是单变量优化的问题,求第二层的最小厚度。
  \item 问题三:该问要考虑第二层和第四层的厚度,是双变量优化问题,但是在现实生活中,第四层不影响研发成本,所以主要优化在于减小第二层的厚度。注意:第三问是判断国赛名次的重要问题,要有自己的创新方法。通过大范围枚举搜索估算两个厚度符合条件的范围,再利用小步长搜索符合条件,得到最优解。
\end{enumerate}
\subsection{模型建立}
\subsubsection{确立热传导方程}
三维等方向均匀介质中的热传导方程为:
$$\frac{\partial T}{\partial t} = \alpha(\frac{\partial^2T}{\partial x^2} + \frac{\partial^2T}{\partial y^2} + \frac{\partial^2T}{\partial z^2}) + \frac{1}{c\rho}q$$
将分层看为无限大平板,并且因为分层中没有热源,所以方程简化为:
$$\frac{\partial T_i}{\partial t} = \alpha \frac{\partial^2T_i}{\partial x^2}$$
\subsubsection{确定边界条件}
热量传递主要有三种形式:热传导、对流和热辐射。需要注意的是,实验室与第一层之间、第四层与假人皮肤之间的热量传递方式为对流方式换热。所以各边界条件为:
\begin{equation*}
  \left\{
  \begin{array}{ll}
    T_i|_{x = x_i} = T_{i+1}|_{x = x_i} & (i = 1, 2, 3)\\
    k_i\frac{\partial T_i}{\partial x}|_{x = x_i} = k_i\frac{\partial T_{i+1}}{\partial x}|_{x = x_{i+1}} & (i = 1, 2, 3)\\
    -k_i\frac{\partial T_1}{\partial x}|_{x = x_0} + h_1T_1|_{x = x_0} = h_1T_s & \text{热对流1} \\
    k_4\frac{\partial T_4}{\partial x}|_{x = x_4} + h_2T_4|_{x = x_4} = h_2T_r & \text{热对流2}\\
    x_i = \sum_{j = 1}^id_i & (i = 1, 2, 3, 4)\\
  \end{array}
  \right.
\end{equation*}

\section{参数求解}
\subsection{模型参数值确定}
使用 $Crank-Nicholson$ 方法进行数值求得 $h_1$,再通过关系式计算得$h_2$。
\subsection{问题二求解}
对附件1给定的 $d_2$ 取值区间进行步长为 $2mm$ 的定步长搜索,确定 $d_2$ 大致范围并进一步减小搜索步长和范围,确定最优厚度。
\subsection{问题三求解}
因为第四层为空气层,第二层为材料层,所以只考虑第二层的成本,所以要先使第二层的厚度尽可能薄。之后采取区域搜索策略,找出所有满足条件的 $(d_2, d_4)$ 组。
\section{模型评价}
\subsection{灵敏度分析}
考虑到实际生产中做工存在偏差,各分层的厚度可能与预期不同,调整各层的厚度,得到厚度变化对皮肤外侧平衡温度的影响。

然后对对流换热系数 $h_1$ 进行调整得到其对第二层最优厚度的影响。
\subsection{优缺点分析}

\section{\LaTeX 学习}
\begin{itemize}
  \item 上下标在正上下方:
  $\sum_i^j\rightarrow \sum\limits_i^j$
  \item 微分算符:\verb|\mathrm{d}t|  $\rightarrow\mathrm{d}t$
  \item 用 minipage 命令排列多张图片:
  \begin{figure}[htbp]
    \centering
    \subfigure[pic1.]{
      \begin{minipage}[t]{0.25\linewidth}
        \centering
        \includegraphics[width = 1in]{figures/drake.png}
      \end{minipage}
    }
    \subfigure[pic2.]{
      \begin{minipage}[t]{0.25\linewidth}
        \centering
        \includegraphics[width = 1in]{figures/drake.png}
      \end{minipage}
    }
    \subfigure[pic3.]{
      \begin{minipage}[t]{0.25\linewidth}
        \centering
        \includegraphics[width = 1in]{figures/drake.png}
      \end{minipage}
    }
    \subfigure[pic4.]{
      \begin{minipage}[t]{0.25\linewidth}
        \centering
        \includegraphics[width = 1in]{figures/drake.png}
      \end{minipage}
    }
    \caption{what }
  \end{figure}
\end{itemize}
\end{document}
