\documentclass[12pt,a4paper]{article}

\usepackage[UTF8]{ctex}
\usepackage{amsmath,amscd,amsbsy,amssymb,latexsym,url,bm,amsthm}
\usepackage{amsfonts}
\usepackage{epsfig,graphicx,subfigure}
\usepackage{hyperref}
\usepackage{listings}
\usepackage[vlined,ruled,linesnumbered]{algorithm2e}
\usepackage{enumitem}
\usepackage{xcolor}
\usepackage{geometry}

%\uppercase\expandafter{\romannumeral1}:% 罗马数字。

\lstset{
language=Matlab,
keywordstyle= \color{blue!70},
commentstyle= \color{red!50!green!50!blue!50},
breaklines
}%设置listing插入语言

\setlength{\parindent}{0em}
\setlength{\parskip}{1em}

\geometry{bottom =3cm}
\newcommand{\textbi}[1]{%
\textbf{\textit{#1}}}

\newcommand{\ncolor}[1]{%
{\color[RGB]{139,117,0}{#1}}}
\newtheorem{theorem}{Theorem}[section]
\newenvironment{solution}{{\noindent \it \textbf{Solution:}}\\}

\title{model LOG}
\author{Yunlong Cheng}

\begin{document}
\maketitle
\section{排队论与随机过程}
\subsection{是什么,解决什么问题}
适用场景:一个有滞留/阻碍现象的系统。

例子
\subsection{知识补充}
\begin{itemize}
  \item \textbf{马尔可夫过程:}随机游走(醉汉),概统书上有。
  \item \textbf{指数分布,泊松分布}
  \item \textbf{技术过程:}表示到事件 $t$ 为止发生的事件的总数。
  \item \textbf{独立增量,平稳增量}
  \item \textbf{泊松过程}
  \item \textbf{排队理论:}生灭过程
  \item \textbf{记号:}X/Y/Z/A/B/C,M:指数分布,k:服务台数量,G:一般分布?
  \item \textbf{基本量和价格方程:}$P_0$
  \item \textbf{稳态概率:}纠错:$P_n = \lim_{t \to \infty}P\{X(t) = n\}$
  \item \textbf{平衡方程:}$()$
\end{itemize}
\subsection{变形}
\begin{itemize}
  \item 有限容量的 M/M/1 型(常用)
  \item 到达率和离开率不是定值
  \item M/M/k型(k个服务台)
\end{itemize}
\subsection{例子}
\begin{enumerate}
  \item 擦鞋店:
  \begin{itemize}
    \item 5个状态
    \item 列出平衡方程,还有概率之和为1
    \item 求解,计算平均顾客数,所耗时间
  \end{itemize}
  \item 串联排队系统:
  \begin{itemize}
    \item 猜测结果,螺旋归纳
  \end{itemize}
  \item 推广:含有 k 条服务线的排队网络(结果用计算机模拟)
\end{enumerate}
\subsection{案例分析}
\begin{enumerate}
  \item 机场安检扫描机安置:\textbf{创新点},算法找改进算法
  \item 高速公路收费站设计:\textbf{创新点},多重选择
  \item 元胞自动机:例子:飞机后排先上前排后上。
\end{enumerate}

\section{评价类模型}
\subsection{打分式评价}
\subsubsection{层次分析法 (AHP)}
基本方法

步骤一:分层
\begin{itemize}
  \item 目标
  \item 准则
  \item 方案
\end{itemize}
步骤二:确定下层因素对上层因素的权重(相互比较)
量表,在5-9个因素范围内

步骤三:计算权向量并且做一致性检验

步骤四:组合权向量,可能要一致性检验(方法自查)

总结:一般之前模型已经建立,所以只需求准则层对目标层的权重

\subsubsection{网络分析法(ANP)}
有双向作用,层之间也有作用。
一般不用。。

\subsubsection{模糊综合分析与评价}
应用场景:多用于

数学表述:
\begin{itemize}
  \item 因素集
  \item 评语集
  \item 评价因素的权重向量
  \item
\end{itemize}

\subsubsection{灰色关联度与评价}
关联度分析

数学表述:
\begin{itemize}
  \item 构造原始数据矩阵
  \item 标准化处理(无量纲化)(重要)
  \begin{itemize}
    \item 初值像
    \item 均值像
    \item 区间值像
  \end{itemize}
  \item 计算灰色关联系数(重要)($\theta$ 灵敏度分析)
  \item 得到灰色关联度矩阵
\end{itemize}

要么选择最好的序列,要么是赛题已给数据。
\subsubsection{数据包络分析(DEA)}
很少用

用线性规划求效率

数学表述:。。。

\subsubsection{主成分分析}
降维方法(将 n 维映射到 k 维),用较少变量解释原数据中大部分差异

主成分:原始数据中蕴含的,能解释大部分变异的新变量。

用于综合评价:
\begin{itemize}
  \item 标准化处理
  \item 计算相关系数矩阵R
  \item 计算特征值和特征向量
  \item 选择 p 个主成分 $(p\le m)$ 计算综合评价值
  \item 计算综合得分
\end{itemize}

非线性,多线性,高阶,稀疏 PCA
实现调库即可

\subsubsection{理想解方法(TOPSIS)}
比较重要

应用场景:多目标规划转单目标规划

相对客观,对数据要求少

两种方法:
\begin{itemize}
  \item ideal point method:唯一
  \item TOPSIS:存在正理想解和负理想解。
\end{itemize}

\subsubsection{熵权法}
idea: 信息熵越小,作用越大

引用场景:样本决策方案书要大于指标数k

\begin{itemize}
  \item 对原始数据归一化,标准化处理
  \item 算各指标的信息熵
  \item 根据信息熵确定权重
  \item 确定最优决策
\end{itemize}

\subsubsection{加权和和加权积}
加权和要对数据规范化

加权积不需要规范化
\subsection{统计类评价}
较小的建模方法,非主旋律

实现:SPSS

基本概念:概统中数理统计

判定系数:是相关系数的平方

\begin{itemize}
  \item 皮尔森相关系数:线性相关
  \item 斯皮尔曼(Spearman)相关系数:单调相关
  \item 肯德尔(Kendall)和谐系数:计算等级变量相关程度的一种相关量。
  \item Wilcoxon 符号秩

  计算
\end{itemize}

方差分析

卡方检验

KS(Kolmogorov-Smirnov)检验:不需要知道数据的分布情况。非参数检验方法。

\section{启发式算法}
\subsection{综述}
在既定的区间找出函数的最大值。

\subsection{模拟退火}
初始温度:
\begin{itemize}
  \item 均匀抽样一组状态,以各状态目标值的方差为初温。
  \item .
  \item .
\end{itemize}

\subsubsection{模拟退火应用}
\begin{itemize}
  \item 求费马点
  \item 找点到 n 条线段的距离和最小
  \item 给定三维空间 n 点,找半径最小的球把这些点全部包围。
\end{itemize}
\subsubsection{调参}
多跑几遍,换随机种子?
\subsubsection{挑战}
\begin{itemize}
  \item 平衡点
  \item TSP
\end{itemize}


\subsection{遗传算法}
\begin{itemize}
  \item 编码
  \item 设定适应度函数
  \item 选择
  \item 交叉
  \item 变异
\end{itemize}

\subsection{禁忌搜索}
\subsection{免疫算法}

\end{document}
