\documentclass[12pt,a4paper]{article}

\title{MCM学习笔记}
\author{llz}
\date{\today}
\usepackage{enumerate}
\usepackage{algorithm2e}
\usepackage{amsmath}
\usepackage[UTF8]{ctex}
\usepackage{gensymb}


\begin{document}
\kaishu
\maketitle

\section{模型学习}

\subsection{粒子群算法PSO}
\subsubsection{基本理论}
从随机解出发,通过迭代寻找最优解,即通过追随当前搜索到的最优值来寻找全局最优。适应度评价解的品质。\\
\indent$D$维目标搜素空间,$m$个粒子组成一个群体,第i个粒子位置表示$X_i=(x_i^1,x_i^2,\dots,x_i^D)$,粒子经历过的最好位置为$P_i=(p_i^1,p_i^2,\dots,p_i^D)$,所有粒子经历过的最好位置记为$P_g=((p_g^1,p_g^2,\dots,p_g^D)$,粒子速度记为
$V_i=(v_i^1,v_i^2,\dots,v_i^D)$。采用下列公式对粒子所在位置不断更新(单位时间1):
$$v_i^d=\omega v^d_i+c_1r_1(p^d_i-x^d_i)+c_2r_2(p^d_g-x^d_i)$$
$$x^d_i=x^d_i+\alpha v^d_i$$
其中$\omega$ 为惯性因子,$c_1$和$c_2$为加速常数,$r_1$和$r_2$为[0,1]内的随机数,$\alpha$ 为约束因子,$v^d_i$限制
在$[-v^d_{max},v^d_{max}]$.
\subsubsection{约束优化}
\begin{enumerate}
\item 罚函数法:将约束优化问题转化为无约束优化问题。
\item 将粒子群的搜索范围都限制在条件约束簇内,即在可行解范围内寻优。
\end{enumerate}
\subsubsection{应用案例:神经网络}

\subsection{模拟退火算法SA}
\subsubsection{基本原理}
温度$T$作为控制参数,目标函数值$f$视为内能$E$,而固体在某温度$T$时的一个状态对应一个解$x_i$。然后算法试图随着控制参数$T$的降低,使目标函数值$f$也逐渐降低,直至趋于全局最小值。
\subsubsection{应用案例}
\begin{enumerate}
\item{旅行商问题TSP:}
\begin{center}
\emph{新解产生:}二变换法、三变换法\\
\emph{Metropolis接受准则:}以目标函数差定义接受概率。
\end{center}
\item{背包问题}
\end{enumerate}

\section{latex学习}

\end{document}
